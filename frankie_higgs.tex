\documentclass[11pt,a4paper]{article}
\usepackage[utf8]{inputenc}
\usepackage{amsmath}
\usepackage{amsfonts}
\usepackage{amssymb}
\usepackage[left=2cm,right=2cm,top=2cm,bottom=2cm]{geometry}


\begin{document}

Speaker:
Frankie Higgs, University of Bath\\

Title:
Connecting two points with random blobs\\

Abstract:
The \emph{random blob} model, or the Boolean model, places balls of radius $r$ centred at the points
of a homogeneous Poisson point process $\mathcal{P}$
in a set $A$
with an intensity $\mu$.
It is closely related to the random geometric graph,
in which two points of $\mathcal{P}$
are joined with an edge if the distance between them
is less than $2r$.
If the relationship between the number of balls and the radii
is such that the average degree of a point,
which is proportional to $\mu r^d$,
converges to a constant
then the model can be thought of as a form of percolation.

Indeed, if $A = \mathbb{R}^d$ or a half-space and $r=1$,
the random blob model is often called \emph{continuum percolation},
parameterised by the intensity.
Many well-known properties of discrete percolation are known
to also hold for continuum percolation,
such as existence of a non-trivial critical intensity
and the existence of a unique unbounded component.

We will discuss recent work
on the two-point connection function
for the Boolean model.
%
If $A$ is a bounded subset of $\mathbb{R}^d$
with a sufficiently smooth boundary,
then given distinct $x, y \in \partial A$,
we prove a limit theorem for the probability
that $x$ and $y$ are joined by a path in the Boolean model.
In particular,
if $\theta_A(\lambda)$ is the probability that the
origin is contained in an unbounded component
in continuum percolation with intensity $\lambda$ in $A$,
then we show
that if $\lambda r^d \to \mu$
as $\lambda \to \infty$,
then the probability that $x$ and $y$ lie in the same component
converges to $\theta_{\mathbb{H}}(\lambda)^2$.

Our proof involves a renormalisation argument
relating certain events for the Boolean model in $A$
explicitly to corresponding events for continuum percolation
in $\mathbb{H}$.

Based on joint work with Mathew Penrose.

\end{document}